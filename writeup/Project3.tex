\documentclass[[11pt]{article}
\usepackage[margin=1in]{geometry}
\usepackage[numbers,sort&compress]{natbib}
\usepackage[bookmarksnumbered,breaklinks,linkcolor=black,citecolor=black,urlcolor=black,bookmarks=true,bookmarksopen=false]{hyperref}
\usepackage[nottoc,numbib]{tocbibind}
\usepackage{graphicx}
\usepackage{textcomp}

\title{GANmons: Sketchable Monsters from Generative Adversarial Networks}
\author{ 
Kristine Lee \\ \texttt{@uic.edu}
\and
Arthur Nishimoto \\ \texttt{@uic.edu}
\and
Xing Li \\ \texttt{@uic.edu}
}

\settocbibname{References}
%\renewcommand{\contentsname}{Table of Contents}

\begin{document}
\maketitle

\begin{abstract}
GANmons generates new images and attributes of 'monsters' using Generative Adversarial Networks (GANs) trained on a dataset of existing creatures and attributes. The goal is to train our neural network on a dataset of existing Pokémon images and attributes (name, type, abilities). After being applied to our GAN, we expect our new GANmons images and attributes to have a similar form to the originals in terms of how appearance and attributes are associated. For example a bird or flying-type GANmon would likely have wings or feathers of some kind. Similarly a grass-type would perhaps have a leafy or plant like appearance. In addition we hope to find interesting results in how names and other attributes such as abilities are associated.
\end{abstract}

\section{Introduction}
[Proper intro about GANs etc.]

The motivation behind this project was work done by Janelle Shane who uses a recurrent neural network (RNN) to randomly generate  Pokémon names, abilities, and attributes based on a training set of existing Pokémon names and abilities \cite{pokemonNN}. Iquanamouth used the output of the RNN to create illustrations of those Pokémon  \cite{pokemonNN-img}. Our work aims to generate new Pokémon, but using a neural network to generate the images rather than by hand.

Our neural network is based on Christopher Hesse's tensorflow implementation of pix2pix by Isola et. al \cite{pix2pix-tensorflow,pix2pix}

\section{Related Work}
Sketchy \cite{sketchy}
Learning What and Where to Draw \cite{whatWhereToDraw}

\section{Implementation}
\subsection{Training Set}
\subsection{Cluster?}
\subsection{Web Interface}
\section{Discussion}


\bibliographystyle{plainnat}
\bibliography{references}

\end{document}



